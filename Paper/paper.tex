%%%%%%%%%%%%%%%%%%%%%%%%%%%%%%%%%%%%%%%%%%%%%%%%%%%%%%%%%%%%%%%%%%
\documentclass[letterpaper, 10pt, conference]{ieeeconf}
\overrideIEEEmargins			% to meet printer requirements
\IEEEoverridecommandlockouts	% to override locked commands

%%%%%%%%%%%%%%%%%%%%%%%%%%%%%%%%%%%%%%%%%%%%%%%%%%%%%%%%%%%%%%%%%%
%REQUIRED PACKAGES%
\usepackage{multirow}
\usepackage{rotating}
\usepackage{float}
\usepackage{caption}
\usepackage{lipsum}
\usepackage{subcaption}
\usepackage{amsmath}
\usepackage{amssymb}
\usepackage{textcomp}
\usepackage{graphicx}
\usepackage{euscript}
\usepackage{ctable}
\graphicspath{{./Figures/}}
\usepackage[nonumberlist,acronym]{glossaries}
\usepackage[hidelinks]{hyperref}

%%%%%%%%%%%%%%%%%%%%%%%%%%%%%%%%%%%%%%%%%%%%%%%%%%%%%%%%%%%%%%%%%%

% correct hyphenation
\hyphenation{temp-orary}

% glossaries
\newacronym{ROS}{ROS}{Robot Operating System}

%%%%%%%%%%%%%%%%%%%%%%%%%%%%%%%%%%%%%%%%%%%%%%%%%%%%%%%%%%%%%%%%%%
\begin{document}

\title{Enhancing Autonomy of Mobile Robots with Behavioral Tree using ROS2}

\author{Lukas Evers$^{*,1}$, Umut Uzunoglu$^{1}$, and Ahmed Hussein$^{1}$ \textit{Senior Member, IEEE}%
    \thanks{$^{*}$ Corresponding author }%
    \thanks{$^{1}$ IAV GmbH, Berlin, Germany \newline
		{\tt\small lukas.evers@iav.de, umut.uzunoglu@iav.de, ahmed.hussein@ieee.org}}%
}

\maketitle
\pagestyle{empty}

%%%%%%%%%%%%%%%%%%%%%%%%%%%%%%%%%%%%%%%%%%%%%%%%%%%%%%%%%%%%%%%%%%

\begin{abstract}

This paper proposes a behavioral planning approach to improve the autonomy of mobile robots using the Robot Operating System (ROS2). The study aims to address the issue of human intervention that is often required during the operation of autonomous mobile robots. The system includes a monitoring system, sensor data storage, and behavior tree design and implementation. The effectiveness of the approach was evaluated using the Gazebo simulator, with the behavior tree handling various scenarios such as sensor failures, collisions, unreachable goals, and low battery state. The simulation results showed a significant improvement in the robot's autonomy and resilience to failures. Additional validation is being performed through real-world experiments to confirm the results.

\end{abstract}

%%%%%%%%%%%%%%%%%%%%%%%%%%%%%%%%%%%%%%%%%%%%%%%%%%%%%%%%%%%%%%%%%%

\section{Introduction}
\label{sec:Introduction}

 Lorem ipsum dolor sit amet, consectetur adipiscing elit. Nam in turpis laoreet magna elementum commodo. Interdum et malesuada fames ac ante ipsum primis in faucibus. Mauris non massa accumsan, molestie velit nec, efficitur leo. Aenean nunc arcu, molestie vitae lectus sit amet, malesuada sagittis lacus. Curabitur fringilla massa ac tellus vestibulum pretium. Sed neque metus, aliquet fringilla tincidunt placerat, vehicula quis justo. Nulla vulputate luctus risus, sed porttitor mauris imperdiet nec. Vestibulum nunc ligula, vestibulum eget nibh sed, volutpat rutrum nisl. Curabitur ultrices nulla urna, et molestie dui rutrum id. Duis faucibus lacinia porttitor. Ut vel commodo nunc~\gls{ROS}~\cite{quigley2009}.

%%%%%%%%%%%%%%%%%%%%%%%%%%%%%%%%%%%%%%%%%%%%%%%%%%%%%%%%%%%%%%%%%%

\section{Related Work}
\label{sec:RelatedWork}

\begin{figure}[ht]
    \centering
    \includegraphics[width=0.9\linewidth]{fig1.png}
    \caption{Figure Caption}
    \label{fig:figureLabel}
\end{figure}

%%%%%%%%%%%%%%%%%%%%%%%%%%%%%%%%%%%%%%%%%%%%%%%%%%%%%%%%%%%%%%%%%%
\section{Proposed Approach}
\label{sec:ProposedApproach}



%%%%%%%%%%%%%%%%%%%%%%%%%%%%%%%%%%%%%%%%%%%%%%%%%%%%%%%%%%%%%%%%%%
\section{Experimental Work}
\label{sec:ExperimentalWork}



%%%%%%%%%%%%%%%%%%%%%%%%%%%%%%%%%%%%%%%%%%%%%%%%%%%%%%%%%%%%%%%%%%

\section{Results and Discussion}
\label{sec:ResultsAndDiscussion}



%%%%%%%%%%%%%%%%%%%%%%%%%%%%%%%%%%%%%%%%%%%%%%%%%%%%%%%%%%%%%%%%%%

\section{Conclusion and Future Recommendations}
\label{sec:Conclusion}


%%%%%%%%%%%%%%%%%%%%%%%%%%%%%%%%%%%%%%%%%%%%%%%%%%%%%%%%%%%%%%%%%%
%\vfill
%\section*{ACKNOWLEDGMENT}

%%%%%%%%%%%%%%%%%%%%%%%%%%%%%%%%%%%%%%%%%%%%%%%%%%%%%%%%%%%%%%%%%%
%\addtolength{\textheight}{-12cm}
%\vspace{10mm}
\bibliographystyle{IEEEtran}
\bibliography{paper}
\end{document}

%%%%%%%%%%%%%%%%%%%%%%%%%%%%%%%%%%%%%%%%%%%%%%%%%%%%%%%%%%%%%%%%%%
