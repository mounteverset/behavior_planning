\chapter{Summary}
\label{cha:diskussion}

What went well? 
This thesis demonstrated a way on how to incorporate advanced behavior planning in ROS2 mobile robots. The goal was to achieve higher levels of autonomy and robustness. The implemented behaviors and the system supervision component have shown to improve the behavior in the simulated scenarios.

However, the results will need to be reproduced on a real mobile robot. Additionaly, the implemented behaviors are a first step to increase the autonomy, but many more behaviors and fallbacks need to be created for the behavior planner. 

The used system architecture is a good base upon which more development can take place without running into problems or limitations in later stages of the development. 

More work is needed to define what is considered to be a safe state or position for the robot to be in after an emergency stop is triggered. Reaching these minimum risk states after emergencies is then the duty of the behavior planner. 

Some behaviors that incorporate rules for autonomous driving, like keeping appropriate distances, right of way, etc., into the planning need to be implemented in future works.

Right now, the robot has no special behaviors for multi-robot system which is a field for many more deliberative behaviors like sharing sensor data with other robots to overcome complete sensor failures. 

What is missing? Further development?


New research questions?



