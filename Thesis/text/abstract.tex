%\section*{Abstract}
\chapter*{Abstract}

%Introduction
This thesis explores ways to improve the autonomy of mobile robots running the Robot Operating System (ROS) with behavioral planning approaches. Autonomous mobile robots often require close human supervision or intervention during their operation. 


%Methods
A system supervision system, a sensor data storage and a behavior tree were designed and implemented. The system was tested on a simulated robot in the Gazebo simulator. Behaviors to handle sensor failures, collisions, unreachable goals, and low battery state. 

%Results
The different test scenarios were artificially induced in a standardized manner to trigger the behaviors to react to the scenarios. The results show an increase in the robot's autonomy and robustness to failures.

%Discussion
The results need to be tested and verified on a real robot, but the increase in the robot's autonomy in the simulation tests are promising an equally positive result. Behavior trees to control and improve the behavior of robots offer the ability to effortlessly add new behaviors to an existing system. This thesis shows a comprehensive way how behavior trees can be implemented to an existing ROS2 system.